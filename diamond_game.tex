% Options for packages loaded elsewhere
\PassOptionsToPackage{unicode}{hyperref}
\PassOptionsToPackage{hyphens}{url}
%
\documentclass[
]{article}
\usepackage{amsmath,amssymb}
\usepackage{lmodern}
\usepackage{iftex}
\ifPDFTeX
  \usepackage[T1]{fontenc}
  \usepackage[utf8]{inputenc}
  \usepackage{textcomp} % provide euro and other symbols
\else % if luatex or xetex
  \usepackage{unicode-math}
  \defaultfontfeatures{Scale=MatchLowercase}
  \defaultfontfeatures[\rmfamily]{Ligatures=TeX,Scale=1}
\fi
% Use upquote if available, for straight quotes in verbatim environments
\IfFileExists{upquote.sty}{\usepackage{upquote}}{}
\IfFileExists{microtype.sty}{% use microtype if available
  \usepackage[]{microtype}
  \UseMicrotypeSet[protrusion]{basicmath} % disable protrusion for tt fonts
}{}
\makeatletter
\@ifundefined{KOMAClassName}{% if non-KOMA class
  \IfFileExists{parskip.sty}{%
    \usepackage{parskip}
  }{% else
    \setlength{\parindent}{0pt}
    \setlength{\parskip}{6pt plus 2pt minus 1pt}}
}{% if KOMA class
  \KOMAoptions{parskip=half}}
\makeatother
\usepackage{xcolor}
\setlength{\emergencystretch}{3em} % prevent overfull lines
\providecommand{\tightlist}{%
  \setlength{\itemsep}{0pt}\setlength{\parskip}{0pt}}
\setcounter{secnumdepth}{-\maxdimen} % remove section numbering
\ifLuaTeX
  \usepackage{selnolig}  % disable illegal ligatures
\fi
\IfFileExists{bookmark.sty}{\usepackage{bookmark}}{\usepackage{hyperref}}
\IfFileExists{xurl.sty}{\usepackage{xurl}}{} % add URL line breaks if available
\urlstyle{same} % disable monospaced font for URLs
\hypersetup{
  hidelinks,
  pdfcreator={LaTeX via pandoc}}

\author{}
\date{}

\begin{document}

Developing strategies for the bidding card game `Diamonds' with GenAI\\
Author: Shivangi Kumari

Introduction:

``Diamonds'' introduces a blend of strategy and bidding, where players
compete to acquire valuable diamond cards for points. The task is to
interact with GenAI and teach it the intricacies of the game in order to
refine its strategic acumen. This report delves into the methodology for
instructing GenAI to play Diamonds and enhance its bidding strategies.
Our ultimate aim is to equip GenAI with the skills to excel at Diamonds
and provide code that can enable a user to play this game.

Problem statement:

Each player gets a suit of cards other than the diamond suit. The
diamond cards are then shuffled and put on auction one by one. All the
players must bid with one of their own cards face down. The banker gives
the diamond card to the highest bid, i.e.~the bid with the most points.
2\textless3\textless4\textless5\textless6\textless7\textless8\textless9\textless T\textless J\textless Q\textless K\textless A
The winning player gets the points of the diamond card to their column
in the table. If there are multiple players that have the highest bid
with the same card, the points from the diamond card are divided equally
among them. The player with the most points wins at the end of the game.

Teaching GenAI the game:

To teach GenAI the intricacies of the card game, we dived into
understanding its rules, mechanics, and strategic nuances that govern
the success of the game.

One of the key insights was the critical role of card valuation in
shaping bidding strategies. We recognized the significance of assigning
point values to each card rank, with higher-ranking cards holding
greater value in the auction process. This enabled it to make informed
bidding decisions based on the relative strengths of the cards and the
potential point gains from acquiring diamond cards.

Furthermore, my conversation with the AI shed light on the dynamic
nature of bidding strategies, which evolve over the course of the game
in response to shifting circumstances and opponent behaviour. We
recognized the need for a delicate balance between bidding aggressively
to secure high-value diamonds and preserving valuable cards for future
auctions. This strategic calculus, rooted in a thorough understanding of
the game's mechanics, served as a guiding principle in GenAI's learning
journey.

Moreover, the importance of player interaction and adaptability in the
game helps to identify recurring trends and strategies that inform
GenAI's decision-making process. By incorporating insights from observed
gameplay and iteratively refining its strategies, GenAI demonstrated a
capacity to adapt and optimize its bidding approach in response to the
evolving game dynamics.

Through in-depth conversation and evaluation, we equipped GenAI with the
knowledge and insights necessary to navigate the complexities of the
game and develop nuanced bidding strategies. This hands-on approach,
grounded in real-world understanding and analysis, laid the groundwork
for GenAI's journey towards the learning of the card game.

Iterating Upon Strategy:

we outline the key strategies employed and the iterative process that
culminated in the final code for the game:

\begin{enumerate}
\def\labelenumi{\arabic{enumi}.}
\tightlist
\item
  Basic Bidding Tactics:

  \begin{itemize}
  \tightlist
  \item
    Initially, we implemented basic bidding tactics, focusing on bid
    selection based solely on the rank of the card from the player's
    suit.
  \end{itemize}
\item
  Card Valuation and Point Assignment:

  \begin{itemize}
  \tightlist
  \item
    Recognizing the importance of card valuation, we introduced a system
    for assigning point values to each card rank, with higher-ranking
    cards receiving higher point values.
  \end{itemize}
\item
  Risk Management and Adaptability:

  \begin{itemize}
  \tightlist
  \item
    Building upon the foundation of card valuation, we integrated risk
    management principles into GenAI's bidding strategies, emphasizing
    the importance of balancing aggressive bidding with prudent card
    preservation.
  \end{itemize}
\item
  Opponent Modeling and Adaptation:

  \begin{itemize}
  \tightlist
  \item
    Leveraging insights from observed player behavior, we implemented
    opponent modeling techniques to analyze bidding patterns and predict
    opponent strategies.
  \end{itemize}
\item
  Endgame Tactics and Optimization:

  \begin{itemize}
  \tightlist
  \item
    As the game progressed and endgame scenarios emerged, we focused on
    developing specialized bidding tactics tailored to maximize point
    gains in the final rounds.
  \end{itemize}
\end{enumerate}

The iterative process of strategy development involved continual
refinement and adjustment, informed by insights gleaned from simulated
gameplay, analysis of opponent behavior, and strategic experimentation.
By iterating upon strategies and incorporating learnings from each
iteration, we navigated towards the final code for the game---a
culmination of strategic sophistication and adaptive decision-making
that embodied the essence of GenAI's mastery of the card game.

Analysis and Conclusion:

The experience of teaching GenAI the card game proved to be a
fascinating journey into the intersection of artificial intelligence,
strategic decision-making, and recreational gaming. Through in-depth
exploration and analysis, we uncovered the complexities inherent in the
game's mechanics and strategies, laying the groundwork for GenAI's
learning and development.

In conclusion, the journey of teaching GenAI the card game represents a
compelling exploration between human expertise and artificial
intelligence capabilities. By combining human ingenuity with AI-driven
strategic acumen, we can unlock new possibilities for enhancing the
competitiveness, enjoyment, and strategic depth of recreational gaming
experiences.

\end{document}